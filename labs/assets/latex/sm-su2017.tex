
\documentclass[11pt]{article}
\usepackage{txfonts}
\usepackage{graphicx}
\usepackage{mygb4e}
\usepackage{solution}

\hidesolutions
\setlength\oddsidemargin{0.01in}
\setlength\topmargin{-1in}
\setlength\textwidth{6.9in}
\setlength\textheight{9.5in} 

\newcommand{\nl}{\mbox{$\langle cr \rangle$}}

\raggedright

% Set some text inside an fbox the full width of the line, with the frame
% sticking out into the margin.
\long\def\framepar#1{\par\noindent\hbox to \textwidth {\hskip-\fboxsep
\fbox{\parbox{\the\textwidth}{#1}}}}

\begin{document}

\begin{center}
{\Large\bf Sample Midterm: CMPT 127 Summer 2017}\\
This material is \copyright Anoop Sarkar 2017. \\
Only students registered for this course are allowed to download and use this material. \\
Use of this material for ``tutoring'' is prohibited.
\end{center}

\noindent{\large\bf Setup}

\begin{itemize}\addtolength{\itemsep}{-0.3\baselineskip}

\item Check that you have a working copy of your gitlab repository.
If not, clone your gitlab repository for this course. It should be
\texttt{CMPT127-1174-username} where \texttt{username} is your SFU
username.

\item Before you do anything new in your repository first pull any
changes to the local repository from the gitlab server by typing
in \texttt{git pull} in your terminal window.

\item \textbf{Create a directory called \texttt{sm} in your repository
(which stands for \textit{sample midterm}).  All the files you write
must be inside this directory.}

\item For each file you add below you must make sure you \texttt{git
add file}, then \texttt{git commit -m "commit message"} and finally
do a \texttt{git push} to send the files to the gitlab server.
Before you leave your terminal make sure you do a \texttt{git
status} to make sure you have committed and pushed your midterm
answers.

\item You are allowed to look at your own lab programming assignments
and the system man pages. Nothing else.

\end{itemize}


\begin{exe}

\ex \textbf{Task 1: Read and print}

\begin{itemize}\addtolength{\itemsep}{-0.3\baselineskip}

\item Write a C program called \texttt{t1.c} with the following requirements.

\item The program should take two command line arguments (using \texttt{argv}): 
the first argument (called \texttt{number}) is a floating point number and the 
second argument (called \texttt{count}) is an
unsigned integer. For example if you compiled your program to a binary \texttt{t1}
then it should be run as follows:

\footnotesize
\begin{verbatim}
cc -Wall -o t1 t1.c -lm
./t1 5.3 10
\end{verbatim}
\normalsize

\item The first floating point number (called \texttt{number}) should be 
converted into a \texttt{double} type by using \texttt{atof()}. 
Then it should be rounded to an integer using the \texttt{round()} 
function. The program should then print out the integer conversion of 
the first argument and print the second argument using the following
format:

\footnotesize
\smallskip
\begin{minipage}{2in}
\begin{verbatim}
./t1 5.3 10
n=5 count=10
\end{verbatim}
\end{minipage}
\begin{minipage}{2in}
\begin{verbatim}
./t1 1.9 15
n=2 count=15
\end{verbatim}
\end{minipage}

\bigskip
\normalsize

\item There should be no extra spaces in the output and end with a newline.

\end{itemize}

\ex \textbf{Task 2: Print bars}

\begin{itemize}\addtolength{\itemsep}{-0.3\baselineskip}

\item Write a C program called \texttt{t2.c} with the following requirements.

\item The program should take two command line arguments (using \texttt{argv}): 
the first argument (\texttt{number}) is a floating point number and
the second argument (\texttt{count}) is an unsigned integer. This
part is identical to \texttt{t1.c}.

\item Convert the first argument (\texttt{number}) into an unsigned
integer. Let us call this integer \texttt{n}.

\item On each line of the output, print out a sequence of \texttt{n}
\texttt{\#} characters followed by a newline.  The number of lines
to print is specified by the second argument \texttt{count} given
to the program. e.g.

\footnotesize
\smallskip
\begin{minipage}{2in}
\begin{verbatim}
cc -Wall -o t2 t2.c -lm
./t2 1.9 3
##
##
##
\end{verbatim}
\end{minipage}
\begin{minipage}{2in}
\begin{verbatim}
./t2 8.1 4
########
########
########
########
\end{verbatim}
\end{minipage}
\begin{minipage}{2in}
\begin{verbatim}
./t2 4 1
####
\end{verbatim}
\end{minipage}

\end{itemize}
\normalsize

\ex \textbf{Task 3: Print a hailstone sequence}

\begin{itemize}\addtolength{\itemsep}{-0.3\baselineskip}

\item Write a C program called \texttt{t3.c} with the following
requirements.

\item The program should take two command line arguments (using
\texttt{argv}): the first argument (\texttt{number}) is a floating
point number and the second argument (\texttt{count}) is an unsigned
integer. This part is identical to \texttt{t1.c}.

\item Convert the first argument (\texttt{number}) into an unsigned
integer. Let us call this integer \texttt{n}.

\item On each line of the output, print out a sequence of \texttt{\#}
characters. The first line contains \texttt{n} \texttt{\#} characters.

\item Each subsequent line should have a sequence of \texttt{\#}
characters equal to the output of the following formula:

\begin{itemize}
\item If \texttt{n} is even, divide it by 2 to give \texttt{n = n/2}.
\item If \texttt{n} is odd, multiply it by 3 and add 1 to give \texttt{n = 3*n+1}
\end{itemize}

\item A number is even if the number mod 2 (\texttt{\%} is the mod operator) gives zero.

\item Continue printing a sequence of \texttt{n} \texttt{\#} characters
using the new value of \texttt{n} computed using the above formula.

\item The program stops printing when the number of lines is equal
to the second argument \texttt{count} given to the program. e.g.

\footnotesize
\smallskip
\begin{minipage}{1.6in}
\begin{verbatim}
cc -Wall -o t3 t3.c -lm
./t3 5 5
#####
################
########
####
##
\end{verbatim}
\end{minipage}
\begin{minipage}{1.3in}
\begin{verbatim}
./t3 5 12
#####
################
########
####
##
#
####
##
#
####
##
#
\end{verbatim}
\end{minipage}
\begin{minipage}{2in}
\begin{verbatim}
./t3 11 6
###########
##################################
#################
####################################################
##########################
#############
\end{verbatim}
\end{minipage}

\normalsize

\end{itemize}

\ex \textbf{Task 4: Print a hailstone sequence (part two)}

\begin{itemize}\addtolength{\itemsep}{-0.3\baselineskip}

\item Write a C program called \texttt{t4.c} with the following
requirements.

\item The program should take two command line arguments (using
\texttt{argv}): the first argument (\texttt{number}) is a floating
point number and the second argument (\texttt{count}) is an unsigned
integer. The program should print out a sequence of \texttt{\#} characters
on each line followed by a newline based on the equation provided
in Task 3. So far this task is identical to \texttt{t3.c}.

\item The difference in this task is that you should stop printing
the sequence of \texttt{\#} characters when you observe the sequence
$4, 2, 1$. If you reach \texttt{count} lines before you observe
$4, 2, 1$ then you should print only \texttt{count} lines. e.g. 

\footnotesize
\smallskip
\begin{minipage}{1.6in}
\begin{verbatim}
cc -Wall -o t4 t4.c -lm
./t4 5 2
#####
################
\end{verbatim}
\end{minipage}
\begin{minipage}{1.3in}
\begin{verbatim}
 ./t4 5 1000
#####
################
########
####
##
#
\end{verbatim}
\end{minipage}
\begin{minipage}{2in}
\begin{verbatim}
./t4 11 1000
###########
##################################
#################
####################################################
##########################
#############
########################################
####################
##########
#####
################
########
####
##
#
\end{verbatim}
\end{minipage}
\bigskip
\normalsize

\item Also try \texttt{./t4 27 1000}. It should print 112 lines of
output. Verify that the last three lines have $4, 2$ and $1$ \texttt{\#}
characters respectively.

\end{itemize}


\end{exe}
\end{document}

